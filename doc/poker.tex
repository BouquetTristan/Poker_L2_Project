\documentclass[12pt,a4paper]{article}
\usepackage[utf8]{inputenc}
\usepackage[french]{babel}
\usepackage[T1]{fontenc}
\usepackage{graphicx}

\begin{document}
\section{Initialisation}
\renewcommand\labelitemi{$\bullet$}
\begin{itemize}
\item{
Déclare un jeu de 52 cartes qui sera modélisé à l'aide de structures pour identifier la couleur et la hauteur des cartes.}
\item{Création des joueurs caractérisé par un pseudonyme, un nombre de jetons, et une main d'un nombre constant de 5 cartes.}
\end{itemize}

\section{Mélange}
\begin{itemize}
\item{
Mélange le jeu en affectant une valeur aléatoire à chaque carte }
\end{itemize}

\section{Distribution}
\begin{itemize}
\item{
Distribue 5 cartes de la pioche à chaque joueur}
\end{itemize}

\section{Tour de jeu}
\begin{itemize}
\item{Chaque tour est caractérisé par 2 phases de tirage\footnote{phase du jeu qui permet à chaque joueur d'échanger n'importe quelle carte(s) de sa main avec la pioche} et 3 phases de mise}
\item{Sens horaire à gauche du croupier\footnote{joueur désigné pour distribuer les cartes} de distribution des cartes et du tour des joueurs }
\item{Le premier joueur doit miser la "petite blind" et le deuxième la "big blind".}
\subsection{ Déroulement phase de mise}
\begin{description}
\item[Début d'une mise : ]{Le joueur après la big blind doit miser la somme minimal  de la big blind et miser au maximum son tapis }
\item[Suite : ]{Le tour de mise se poursuit jusqu'à que tous les joueurs aient misé le même nombre de jeton}
\end{description}
\subsection{Tirage}
Les joueurs échangent à tour de rôle, dans le même sens de jeu de la mise, un nombre de carte compris entre 0 et 4 au maximum.
\end{itemize}
\section{Abattage}
Les joueurs présentent leurs cartes et le joueur avec la meilleure combinaison remporte le pot. En cas d'égalité, le pot est partagé et réparti équitablement entre les joueurs à égalité.
\section{Répartition des tâches}
\begin{description}
\item[Romain : ]{Création et initialisation des objets (carte, jeu de carte, joueur, table de jeu), mélange et distribution des cartes, couche graphique en SDL}
\item[Tristan : ]{Abattage (analyse des combinaison des mains des joueurs, détermination du gagnant, et répartition des gains}
\item[William :]  {Tour de jeu (système de mise, tirage)}
\item[OPTIONEL : ]{Jeu multijoueur en réseau local}
\end{description}

\section{Logigramme provisoire}
\includegraphics[scale=0.69]{logigramme}

\end{document}
